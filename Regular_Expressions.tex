% Compile by directly running 
%   pdflatex Introduction_to_Python.tex
%
%
%
%
%
\documentclass{beamer}
\usetheme{Copenhagen}
\usepackage[utf8]{inputenc}
\usepackage{times}  % fonts are up to you
\usepackage{graphics}
\usepackage{amsmath}
\usepackage{media9}
\usepackage{hyperref}
\usepackage{psfrag}
\usepackage{pdfpages}
\usepackage{listings}
%\usepackage[hyphens]{url}

% https://tex.stackexchange.com/questions/116534/lstlisting-line-wrapping
%\lstset{
%  basicstyle=\ttfamily,
%  columns=fullflexible,
%  frame=single,
%  breaklines=true,
%  postbreak=\mbox{\textcolor{red}{$\hookrightarrow$}\space},
%}

\hypersetup{%
  colorlinks=true,% hyperlinks will be black
  linkbordercolor=red,% hyperlink borders will be red
  pdfborderstyle={/S/U/W 1}% border style will be underline of width 1pt
}

\setbeamertemplate{bibliography item}[text]
\newcommand{\customcite}[1]{\citeauthor{#1}, \citeyear{#1}}
\newcommand\smallFont{\fontsize{8}{7.2}\selectfont}   %Change font size.
\newcommand\mCite[1]{[\cite{#1}, \citetitle{#1}]}  %Prints name and title
\newcommand\FrameText[1]{
\begin{textblock*}{\paperwidth}(0pt,\textheight)
	\vspace{1.0cm}
    \raggedleft \smallFont #1
\end{textblock*}}

%Get rid of ugly copenhagen default symbol for enumerate
\setbeamertemplate{enumerate items}[default]   
\setbeamertemplate{itemize item}[circle]
\setbeamertemplate{itemize subitem}{\raise1.5pt\hbox{\donotcoloroutermaths$\blacktriangleright$}}
\setbeamertemplate{itemize subsubitem}{\raise1.5pt\hbox{$\star$}}


% Create code text
% https://tex.stackexchange.com/questions/65291/code-snippet-in-text
\definecolor{codegray}{gray}{0.9}
\newcommand{\code}[1]{\colorbox{codegray}{\texttt{#1}}}



%Information to be included in the title page:
\title{Regular Expressions}
\author{Ali Snedden}
\institute{Nationwide Children's Hospital}
\date{October 18, 2019}
 
 
 
\begin{document}
 
\frame{\titlepage}

% Recurring Outline
\AtBeginSection[]  % "Beamer, do the following at the start of every section"
{
\begin{frame}<beamer> 
\frametitle{Outline} % make a frame titled "Outline"
\tableofcontents[currentsection]  % show TOC and highlight current section
\end{frame}
}

\begin{frame}
\frametitle{Get This Presentation}
How to get this presentation:
\bigskip
\href{https://github.com/astrophys/Regular\_Expressions}{https://github.com/astrophys/Regular\_Expressions}
\end{frame}


\begin{frame}
\frametitle{How to Connect}
Windows:
{\scriptsize
\begin{itemize}
    \item Open PuTTY
    \item Click Window Session $\Rightarrow$ Host Name field : 10.70.250.101
    \item Click on Connection $\Rightarrow$ SSH $\Rightarrow$ X11, check the box next to ``Enable X11 forwarding"
    \item Click ``Open" to log in.
    \item Enter username and password
    \item Start Xming on Windows
\end{itemize}
}
\bigskip

Mac:
{\scriptsize
\begin{itemize}
    \item Open Terminal (Finder $\Rightarrow$ Utilities $\Rightarrow$ Terminal)
    \item \code{ssh -X username@10.70.250.101}
\end{itemize}
}
\end{frame}

\begin{frame}
\frametitle{What are Regular Expressions?}
Definition
\begin{itemize}
    \item A sequence of characters that define a search pattern
\end{itemize}
\pause
Uses
\begin{itemize}
    \item Searching large text files
    \pause
    \item Complex search and replace in text editors
    \pause
    \item Parsing output in linux pipelines
\end{itemize}
\pause
Aliases:
\begin{itemize}
    \item regex (pl. regexes)
    \pause
    \item regexp
\end{itemize}
\pause
Available in :
\begin{itemize}
    \item \code{grep}, \code{egrep}
    \pause
    \item \code{vim,emacs}
    \pause
    \item \code{awk,sed}
    \pause
    \item \code{Python,Perl,R}
\end{itemize}
\end{frame}


\begin{frame}
\frametitle{Regular Expressions}
Two types of characters
\begin{itemize}
    \item Metacharacter : character with a special meaning
    \begin{itemize}
        \item \code{.} : wildcard 
        \pause
        \item \code{[ ]}: variable matches
        \pause
        \item \code{( )}: subexpression which can be recalled later (used in \code{vim} search and replace
        \pause
        \item \code{\{ \}}: denotes number of matches
        \pause
        \item \code{\^{}} : anchor regex at beginning of line
        \pause
        \item \code{\$} : anchor regex at end of line
    \end{itemize}
    \pause
    \item Literal : character with a special meaning
\end{itemize}
\end{frame}

\begin{frame}
\frametitle{Gotchya's}
Syntax isn't 100\% consistent across regex engines
\begin{itemize}
    \item Common issue : often need to escape metacharacters and which ones need escaped often differs.
    \pause
    \item I often find myself doing little tests to ensure I know how the engine works.
\end{itemize}
\end{frame}

\begin{frame}
\frametitle{Metacharacter : ``."}
Wildcard
\bigskip
Compare :
\begin{itemize}
    \item \code{grep -E Texas data/Democracy\_in\_America.txt}
    \pause
    \item \code{grep -E .exas data/Democracy\_in\_America.txt}
\end{itemize}
\end{frame}


\begin{frame}
\frametitle{Metacharacter : ``\^{}"}
Anchors match to beginning of line
\bigskip

Compare :
\begin{itemize}
    \item \code{grep -E state data/Democracy\_in\_America.txt}
    \pause
    \item \code{grep -E \^{}state data/Democracy\_in\_America.txt}
\end{itemize}
\end{frame}

\begin{frame}
\frametitle{Metacharacter : ``\$"}
Anchors match to end of line
\bigskip

Compare :
\begin{itemize}
    \item \code{grep -E state data/Democracy\_in\_America.txt}
    \pause
    \item \code{grep -E state\$ data/Democracy\_in\_America.txt}
\end{itemize}
\end{frame}

\begin{frame}
\frametitle{Metacharacter : ``[ ]"}
Includes characters contained. 
\bigskip

Compare :
\begin{itemize}
    \item \code{grep -E disq data/Democracy\_in\_America.txt}
    \pause
    \item \code{grep -E dis[qo] data/Democracy\_in\_America.txt}
    \pause
    \item \code{grep -E dis[a-p] data/Democracy\_in\_America.txt \# Lowercase Letters a-p}   
\end{itemize}
\end{frame}

\begin{frame}
\frametitle{Metacharacter : ``[\^{} ]"}
Excludes characters contained. 
\bigskip

Compare :
\begin{itemize}
    \item \code{grep -E disq data/Democracy\_in\_America.txt}
    \pause
    \item \code{grep -E dis[\^{}tqo] data/Democracy\_in\_America.txt}
\end{itemize}
\end{frame}

\begin{frame}
\frametitle{Metacharacter : ``\{m,n\}"}
Match preceding element at least \code{m}, not more than \code{n} times.
\bigskip

Compare :
\begin{itemize}
    \item \code{grep -E as\{2\}a data/Democracy\_in\_America.txt}
    \pause
    \item \code{grep -E [as]\{4,6\} data/Democracy\_in\_America.txt}
    \pause
    \item \code{grep -E [as]\{5\} data/Democracy\_in\_America.txt}
\end{itemize}
\end{frame}

\begin{frame}[fragile]
\frametitle{Metacharacter : ( ) }
\begingroup
Denotes sub expression that can be recalled later.
\bigskip

In vim consider the corrupted file:
\begin{lstlisting}[backgroundcolor = \color{codegray}, language = Python, showstringspaces=false, breaklines=true]
:%s/\([0-9]\.[0-9]\{3\}e[+-][0-9]\{2\}\)\([0-9]\.[0-9]\{3\}e[+-]\)/\1 \2/gc
\end{lstlisting}
\endgroup
\end{frame}




\begin{frame}
\frametitle{Homework}
\begin{itemize}
    \item Find all lines where the 2nd and 3rd characters are either an 'a' or an 's' and the last word is 'pioneers'
    \pause
    \item Find all the lines where either '[Ss]lavery' and freedom occur simultaneously.
    \pause
    \item Count the number of times that either 'Indian', 'white' and 'black' occur in the document (hint : look at man page for \code{grep})
    \pause
    \item Print out all lines where 'war' or 'wars' is followed by punctuation (i.e. ',', '.', '-')
\end{itemize}
\end{frame}





\begin{frame}
\frametitle{Resources}

\href{https://www.regular-expressions.info/}{https://www.regular-expressions.info}

\href{https://stackoverflow.com/}{Stack Overflow}

\end{frame}


\begin{frame}
\frametitle{Answers}
\begin{itemize}
    \item \code{grep -E \^{}.[as]\{2,10\}.\{1,100\}pioneers\$ data/Democracy\_in\_America.txt}
    \pause
    \item \code{grep -E [Ss]lavery data/Democracy\_in\_America.txt  | grep freedom}
    \pause
    \item \code{grep -E -c Indian data/Democracy\_in\_America.txt}
    \pause
    \item \code{grep -E -c war[\^{}a-r,t-z\ ][\^{}\ ] data/Democracy\_in\_America.txt}
\end{itemize}
\end{frame}




\end{document}
